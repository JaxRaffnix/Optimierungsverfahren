\section{Lösung mit Bayesian Search}


\begin{frame}{Die Idee von Bayesian Search}
    \textbf{Modell:} $m\,(Epochen,\,Lernrate\,\rightarrow\,Accuracy)$
    \begin{itemize}
        \item Näherung der Zielfunktion $f\,(Epochen,\,Lernrate)$
        \item schnell auswertbar
        \item Wird nach jeder neuen Auswertung von $f$ weiter verbessert
    \end{itemize}

    \vspace*{1em}
    \textbf{Exploitation:} Vorhandenes Wissen nutzen
    \begin{itemize}
    \item Wähle Punkte, an denen $m$ eine hohe Accuracy vorhersagt
    \item $f$ an dieser Stelle berechnen $\rightarrow$ vermutlich neuer Bestwert
\end{itemize}

    \vspace*{1em}
    \textbf{Exploration:} Unbekannte Bereiche ausprobieren
    \begin{itemize}
        \item Wähle Punkte, an denen $m$ und $f$ stark voneinander abweichen könnten
        \item Verbessert das Modell und entdeckt neue gute Bereiche
    \end{itemize}
\end{frame}

%___________________________________________________________________
\begin{frame}{Animation zu Bayesian Search}
\begin{columns}[T, totalwidth=\textwidth]

    % Linke Spalte: Animation
    \begin{column}{0.55\textwidth}
        \centering
        \animategraphics[loop, autoplay, controls, width=\linewidth, height=\imageheight, keepaspectratio]{0.5}{videos/frame_}{000}{006}
    \end{column}

    % Rechte Spalte: Caption und Beschreibung
    \begin{column}{0.45\textwidth}
    \captionof{figure}{Bayes'sche Scoreoptimierung für einen Random-Forest-Klassifizierer}
    \imagesource{\cite{bayes_gif}}
    \imagelegend{
        \scriptsize \\
        \begin{itemize}[leftmargin=0pt]
            \item \textit{x-Achse:} Parameter des Random-Forest-Klassifizierers. 
            \item \textit{Schwarz:} Zielfunktion. 
            \item \textit{Lila:} Modell mit Unsicherheitsbereich. 
            \item \textit{Expected Improvement:} Erwarteter Zugewinn gegenüber dem aktuellen Bestwert.  
            \item \textit{Upper Confidence Bound:} Finde vielversprechende, aber unerkundete Bereiche. 
            \item \textit{Probability of Improvement:} Wahrscheinlichkeit, einen besseren Punkt zu finden.
        \end{itemize}   
    }
\end{column}

\end{columns}
\end{frame}

%___________________________________________________________________
\begin{frame}{Eregbnisse mit Bayesian Search}
    \begin{figure}
        \centering
        \includegraphics[width=\imagewidth, height=\imageheight, keepaspectratio]{contour_bayesian}
        \caption{Accuracy Ergebnisse für Bayesian Search}
    \end{figure}
\end{frame}