\section{Bayesian Search}

%___________________________________________________________________
\begin{frame}{Animation zu Bayesian Search}
\begin{columns}[T, totalwidth=\textwidth]

    % -------------------------------------------------------------
    % Linke Spalte: Animation
    \begin{column}{0.55\textwidth}
        \centering
        \animategraphics[loop, autoplay, controls, width=\linewidth, height=\imageheight, keepaspectratio]{0.5}{videos/frame_}{000}{006}
    \end{column}

    % -------------------------------------------------------------
    % Rechte Spalte: Caption und Beschreibung
    \begin{column}{0.45\textwidth}
    \captionof{figure}{Bayes'sche Optimierung eines Scores für einen Random-Forest-Klassifizierer}
    \imagesource{\cite{bayes_gif}}
    \imagelegend{
        \scriptsize
        \textit{x-Achse:} Parameter des Random-Forest-Klassifizierers. \textit{Schwarz:} Zielfunktion. \textit{Lila:} Modellierte Funktion mit Unsicherheitsbereich ±1 Standardabweichung. \\ 
        \textit{Expected Improvement:} Erwarteter Zugewinn gegenüber dem aktuellen Bestwert. \\ 
        \textit{Upper Confidence Bound:} Suche vielversprechende, aber unerkundete Bereiche. \\
        \textit{Probability of Improvement:} Wahrscheinlichkeit, dass ein neuer Punkt besser ist als der bisherige Bestwert.
    }
\end{column}

\end{columns}
\end{frame}